\documentclass[conference]{IEEEtran}
\setlength{\parskip}{0.75em}

\hyphenation{op-tical net-works semi-conduc-tor}

\begin{document}

\title{Challenges of Self Organization\\in Agile Projects}


\author{\IEEEauthorblockN{Andrew Hughson}
\IEEEauthorblockA{Dept. of Electrical and Computer Engineering\\
University of Auckland\\
ahug048, 1546814}}


\maketitle


\begin{abstract}
Agile is a thingy
\end{abstract}


\section{Introduction} This is the introduction test


\section{Related Work}
RESEARCH STUFF


\section{Project Experience}

The project we undertook from Orion Health was a streamlined check in system
which we dubbed ``Vital Stats Manager'' or ``VSM'' which is intended to replace
paper forms when arriving at a hospital check in. The system consists of four
major components, three of which we tackled over four sprints. These components
are the VSM Android Application, the VSM NFC Receiver, the VSM Server and API,
and the VSM Web Application.

The VSM Android Application is intended to be distributed and installed on
patient's Android devices. It consists of a form which replaces the paper form
normally filled out when checking in to a hospital. This form allows patients to
enter their vital medical information, such as their height, weight, allergies,
etc.

The VSM NFC Receiver is the component of the system which was not fully
developed over the four sprints of the project due to limitations in the
availability of required NFC hardware. For the purposes of the project a mock
receiver was developed as an Android application, though the user interface was
not a focus. The role of this component is to receive patient check ins. At the
reception of each hospital department it is envisioned that there will be one of
these receivers which a patient can touch with their Android device with the VSM
Android Application installed. By touching the receiver they transmit their
vital information to the hospital and are checked in.

The VSM Server and API component is the component which receives, stores, and
exposes patient check ins. When a patient checks in by transmitting their vital
information over NFC to the receiver, the receiver uploads the received data to
the server which stores it in a database. This information is then exposed via a
RESTful API for consumption by other components of the system.

The VSM Web Application is the main consumer of the data exposed by the VSM
Server and API. It is intended that receptionists and nurses can see recent
check ins to a department and review and edit patient vital information as
patients check in to the hospital. It provides functions to search, filter, edit
and delete patients.

The team for this project consisted of six members; myself, Michael Little,
Jourdan Harvey, Dave Carpenter, Andrew Luey, and Thomas Lugnet. This team was
self assembled before the project requirements were available. Each member was
familiar with the others and had differing degrees of software development
experience.

After receiving the project description and before our first meeting with our
customer representatives we decided to work on a prototype of the system based
on our understanding of the requirements. This gave the team an opportunity to
research and familiarize themselves with the technologies which would be used to
develop the final working system.

At our first meeting with our customer representatives we clarified the project
requirements and realised our understanding was not complete. Fortunately our
prototypes of the system components were modifiable to the point where this was
not a large concern and our technology choices were not affected. Here we see an
example of the way Agile teams can react more easily to changes in requirements,
due to the short iterations and frequent customer interaction our software was
able to adapt easily.

After gathering the requirements from our first meeting with our customer
representatives we self organised into pairs roughly mapping to the three
components which we were to implement. Particular members of our team had more
experience in different areas of development. This was important with regard to
the manner in which we self organised. Dave and myself had experience in back
end development relevant to the Server and API, Michael had experience in
Android development, and Dave, Jourdan and I had experience in web front end
development.

To encourage our team members to become ``masters of all trades'' and avoid too
much specialisation we organised pairs so that a member with experience in the
area of the system they were developing was paired with a member with less or no
experience in that area. By practising pair programming members were able to
learn about the technologies and the implementation details of the part of the
system on which they were working. In this manner we avoided any one team member
becoming a specialist, though there was a certain level of skill segregation
amongst the team. This can be attributed to the time constraints of the project
and the need to learn new technologies. We were unable to sacrifice the time to
let all users spend time learning each of the myriad technologies employed
across the three components of our system.

Three pairs ended up forming - myself and Andrew Luey worked on the VSM Server
and API. I had experience in developing RESTful APIs backed by databases similar
to the way this component was going to work. Michael had previous Android
development experience and so Jourdan and Michael worked on the VSM Android
Application. Dave had some limited experience in web development and so Dave and
Thomas worked on the VSM Web Application.

Despite certain members having relevant experience, the specific technologies we
chose were not necessarily familiar. Those members with experience could more
easily pick up these new technologies and therefore expedite the learning of
those less experienced members. For example, we used an ORM called SQLAlchemy to
interface with our database in the VSM Server component. I was able to quickly
pick up this technology from my previous experience with different ORMs and in
the process teach Luey the important aspects necessary for the development of
our system.

Over the development of the project we had roughly weekly meetings with our
customer representatives from Orion Health - that is two meetings per sprint. As
Agile practitioners themselves they acted as mentors as well as customers.

In their role as mentors they advocated certain Agile practices to us -
especially test driven development. We managed to incorporate this technique
more and more as our software stabilised and we became more familiar with the
technologies we were using.

As customers our representatives at Orion provided us with the requirements for
the project. As the project progressed and we completed the original user
stories and met milestones they would add requirements to the project. Here we
see another example of Agile methodologies and the way requirements can change
and planning can occur at late stages of the project, as opposed to more
prescriptive methodologies in which changes may be more difficult to
incorporate.

Some examples of late requirements added by our customers were the additions of
department log ons and breadcrumbs. The customers asked for the ability to
separate patients by departments and let nurses and administrative staff see
only those patients from specific departments. This requirement was added in the
fourth sprint of the project after we had become familiar with our code and the
technologies we were using. This gave us an opportunity to practice TDD. The
main change had to occur in the server where we needed to change the way
patients were accessed to limit them to departments. We were able to add tests
which would verify the correctness of our implementation and then implement the
changes. From this we saw the value of the red-green-refactor cycle.

The second requirement - the addition of breadcrumbs to the web interface - was
also introduced in the fourth sprint. At this point we were able to become more
cross functional as a team as we had overcome the learning phase. As a result I
was able to work on this requirement which was outside of the VSM Server
component I had been mainly responsible for up to this point. Given more time I
can see how this could have occurred in other places in the project as work
would run out in specific areas and members would be required to undertake tasks
in unfamiliar areas of the code base.


\section{Discussion}


\section{Implications for Practice}


\section{Conclusion}


\section*{Acknowledgement}


\begin{thebibliography}{1}

\bibitem{IEEEhowto:kopka}
H.~Kopka and P.~W. Daly, \emph{A Guide to \LaTeX}, 3rd~ed.\hskip 1em plus
  0.5em minus 0.4em\relax Harlow, England: Addison-Wesley, 1999.

\end{thebibliography}


\end{document}
